% Template for SGN-3507 papers; to be used with:
%          spconf.sty   - ICASSP/ICIP LaTeX style file
%          IEEEtran.bst - IEEE bibliography style file

% Originally created for MCSP 20xx papers;
% Created:  Sep 2006 - Riku Uusikartano -- riku.uusikartano@tut.fi
% Modified: Sep 2007 - Riku Uusikartano -- riku.uusikartano@tut.fi
% Modified: Oct 2009 - Riku Uusikartano -- riku.uusikartano@tut.fi
% Modified: Oct 2010 - Jukka Suhonen -- jukka.suhonen@tut.fi
% Modified: Oct 2011 - Francescantonio Della Rosa -- francescantonio.dellarosa@tut.fi
% Modified: Nov 2011 - Jukka Suhonen -- jukka.suhonen@tut.fi
% Modified for SGN-3507 March 2013 - Juha Pajula -- juha.pajula(a)tut.fi
% Modified for SGN-16006 Jan 2014 - Hanna Silen -- hanna.silen(a)tut.fi

% --------------------------------------------------------------------------
\documentclass{article}

% The amsmath and epsfig packages greatly simplify the process of adding
% equations and figures to the document, and thus their use is highly
% recommended.
% ------------
\usepackage{spconf,amsmath,epsfig}
\usepackage{subcaption}

% Title.
% ------
\title{Smile Recognition}

\name{Jesper Granat, Peter Todorov
%\thanks{Insert sponsor acknowledgments (where necessary) here.}
}
%
\address{jesper.granat@tuni.fi, peter.todorov@tuni.fi}

% Hyphenation (hyphenate all names and non-english words here).
% -------------------------------------------------------------
\hyphenation{Tam-pe-re micro-soft}

\begin{document}

\maketitle
\sloppy

% Abstract.
% ---------
\begin{abstract}
In this study, a convolutional neural network was trained to recognize
smile from an image. The training and deployment was done using the
Keras framework on top of the Tensorflow framework. The training data
was 4 000 images of smiling and non-smiling faces. The accuracy on
this dataset was 86,XXX \%.
\end{abstract}

% First section, often named as Introduction.
% -------------------------------------------
\section{Introduction}
\label{sec:intro}
Since their introduction [REFERENCE], Convolutional Neural Networks
(CNNs) have been widely used in solving various image processing
problems, such as classification, object recognition and image
enhancement [REFERENCE]. In this study, a custom design CNN is trained
for real-time smile recognition. Real-time in this context refers to
inference speed where the network can be run on live video feed in
over 20 fps on a laptop with GPU\@.

% Optional section.
% ---------------
\section{Background}
\label{sec:background}
% Third section
\section{Experiments}

% Forth section
\section{Results}
\label{sec:results}

% Fourth section
\section{Conclusion}
\label{sec:conclusion}

\small

% IEEEtran is a LaTeX style file defining the reference formatting.
% -----------------------------------------------------------------
\bibliographystyle{IEEEtran}

% IEEEabrv is a LaTeX style file defining the abbreviations of different
% journals and conferences. mcsp_refs contains the actual reference data
% from which the references are selected into the paper using \cite{}.
% ----------------------------------------------------------------------

% Comment of following line if bibliography is not needed
%\bibliography{mcsp_refs}

% ---------------------------------------------------------------------------
\vfill\pagebreak

\end{document}
